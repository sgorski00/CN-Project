\section{Nginx}
Kolejnym kluczowym elementem aplikacji webowej jest serwer HTTP. Do obsługi ruchu HTTP użyty został serwer \verb|Nginx|, który został skonfigurowany do prekazywania żądań do aplikacji PHP. Użyta została wersja \verb|latest|, co oznacz najnowszą wersję z możliwych. Zabieg ten ma na celu ciągłe aktualizacje, które zwiększą bezpieczeństwo aplikacji. Mimo zmieniającej się wersji serwera \verb|Nginx|, nie powinno tworzyć to problemów. Serwer Nginx działa na porcie 80, czyli na nim aplikacja będzie uruchamiana.

\subsection{docker-compose.yml}
W tej sekcji konfiguracja jest prosta. Wszystkie klucze zostały omówione w poprzednich podrozdziałach dotyczących innych usług.

\begin{lstlisting}[language=yaml, caption={Konfiguracja kontenera Nginx w pliku docker-compose.yml}, label={lst:docker_compose_nginx}, numbers=left, frame=single]
    nginx:
        image: nginx:latest
        container_name: nginx-container
        ports:
            - "80:80"
        volumes:
            - ./environment/dev/nginx/nginx.conf:/etc/nginx/nginx.conf:ro
            - ./:/var/www
        networks:
            - cn-network
        depends_on:
            - php
\end{lstlisting}

\textit{Warto zwrócić uwagę na podpięcie pliku konfiguracyjnego do kontenera serwera HTTP. Znacznik :ro oznacza read-only (tylko do odczytu). Opis pliku znajduje się w rozdziale 3.3.2}

\subsection{nginx.conf}
Plik \verb|nginx.conf| definiuje w jaki sposób \verb|Nginx| ma obsługiwać przychodzące żądania. W przypadku tego proejktu jest podzielony na dwie sekcje - \verb|events| - listing \ref{lst:nginx_config_events} i \verb|http| - listing \ref{lst:nginx_config_http}.

\begin{lstlisting}[style=nginx, caption={Plik konfiguracyjny nginx.conf - sekcja events}, label={lst:nginx_config_events}]
events {
    worker_connections 1024;
}
\end{lstlisting}

W tym przypadku została określona liczba jednoczesnych połączeń na 1024.

\newpage
\begin{lstlisting}[style=nginx, caption={Plik konfiguracyjny nginx.conf - sekcja http}, label={lst:nginx_config_http}]
http {
    server {
        listen 80;
        server_name localhost;

        root /var/www/public;
        index index.php index.html;

        location / {
            try_files $uri $uri/ /index.php?$query_string;
        }

        location ~ \.php$ {
            include fastcgi_params;
            fastcgi_pass php-container:9000;
            fastcgi_param SCRIPT_FILENAME $document_root$fastcgi_script_name;
            fastcgi_intercept_errors on;
        }

        location ~ /\.ht {
            deny all;
        }
    }
}
\end{lstlisting}

Ta sekcja jest bardziej rozbudowana. Mówi między innymi o tym, że serwer \verb|Nginx| będzie nasłuchiwał na serwerze o nazwie localhost, na porcie 80. Katalog okreslony jako główny to /ver/www/public. Domyślnym plikiem indexowym jest plik \verb|index| z rozszerzeniem \verb|php| lub \verb|html|.

Następnie jest \verb|location /| - pole to określa w jaki sposób serwer ma obsłużyć żądanie HTTP wysłane na stronę. \verb|Nginx| spróbuje znaleźć plik lub katalog odpowiadający żądanej ścieżce. Jeśli ścieżka nie istnieje, pzekieruje żądanie do \verb|index.php| wraz z parametrami \verb|$query_string|.

Najdłuższa sekcja na liście to \verb|location ~ \.php$| - określa ona regułę obsługi wszystkich plików PHP. Skonfigurowane zostały tutaj niezbędne parametry do komunikacji z PHP. Najbardziej interesującym nas elementem jest \verb|fastcgi_pass| - ustawiamy tutaj, gdzie \verb|Nginx| ma szukać serwera PHP
\textit{Wpisana musi być nazwa kontenera, na którym jest uruchomiony PHP wraz z portem!}

Na koniec zablokowany jest dostęp do wszystkich plików, kórych nazwa zaczyna się od .ht (np. \verb|.htaccess|). Jest to istotne pod względem bezpieczeństwa aplikacji.