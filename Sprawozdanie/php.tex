\section{PHP}
PHP jest serwerowym językiem programowania kluczowym dla działania aplikacji, w tym projekcie skonfigurowanym z frameworkiem \verb|Laravel|, który usprawnia tworzenie aplikacji webowych. Użyto wersji PHP 8.2-fpm, zgodnej z Laravel 9 i nowszymi, zapewniającej wysoką wydajność dzięki FastCGI Process Manager (FPM).

Kontener PHP wzbogacono o dodatkowe zależności, takie jak Composer, zarządzający pakietami, oraz Redis, używany jako pamięć podręczna. Ten kontener obsługuje logikę aplikacji i komunikuje się z innymi usługami.
\subsection{docker-compose.yml}
Plik \verb|docker-compose.yml| definiuje konfigurację kontenera PHP.

\newpage
\begin{lstlisting}[language=yaml, caption={Konfiguracja kontenera php w pliku docker-compose.yml}, label={lst:docker_compose_php}, numbers=left, frame=single]
    php:
        build:
            context: .
            dockerfile: /environment/dev/php/Dockerfile
        container_name: php-container
        working_dir: /var/www
        volumes:
            - ./:/var/www
        expose:
            - "9000:9000"
        networks:
            - cn-network
        depends_on:
            - mysql
            - redis
\end{lstlisting}

W listingu \ref{lst:docker_compose_php} pokazana jest część pliku \verb|docker-compose| odpowiedzialna za utworzenie kontenera PHP. Można zauważyć, że sekcja ta dzieli się na kilka części.
\begin{itemize}
    \item \verb|build| - komenda używana do tworzenia obrazu kontenera na podstawie plików źródłowych.
    \begin{itemize}
        \item \verb|context| - miejsce wywołania pliku (\textit{. - aktualna ścieżka}).
        \item \verb|dockerfile| - lokalizacja pliku Dockerfile. W pliku Dockerfile został utworzony niestandardowy obraz na podstawie istniejącego już obrazu.
    \end{itemize}
    \item \verb|container_name| - nazwa kontenera - pole nieobowiązkowe, system sam może nadać nazwę, ale tutaj użytkownik posiada większą kontrolę.
    \item \verb|working_dir| - określa katalog roboczy, w którym kontener rozpocznie pracę.
    \item \verb|volumes| - określa jakie katalogi i pliki zostaną zamontowane do kontera. Wszystkie pliki źródłowe aplikacji zostają przeniesione do katalogu /var/www w kontenerze PHP.
    \item \verb|expose| - określa porty, które kontener udostępnia innym kontenerom wewnątrz sieci (w tym przypadku inne kontenry będą mogły odnaleźć PHP na porcie 9000 w sieci cn-network).
    \item \verb|networks| - sieci, w których kontener będzie pracował (w przypadku tego projektu wszystkie kontenery będą pracowały w jednej sieci, żeby miały do siebie swobodny dostęp).
    \item \verb|depends_on| - komenda, w której można ustawić "priorytety" kontenerów. W tym przypadku kontener PHP zależy od kontenerów: \verb|mysql| i \verb|redis|, co oznacza, że PHP nie zostanie uruchomiony, dopóki wymienione kontenery nie zostaną uruchomione.
\end{itemize}

\subsection{Dockerfile}
Plik Dockerfile pozwala na utworzenie niestandardowego obrazu na podstawie obrazu pobranego z \verb|Docker Hub|. Do tego pliku odwołała się usługa php w sekcji \verb|build|. W pliku tym ważne jest wskazanie z jakiego obrazu bazowego należy korzystać.

\newpage
\begin{lstlisting}[style=dockerfile, caption={Konfiguracja obrazu php w pliku Dockerilfe}, label={lst:dockerfile}]
FROM php:8.2-fpm

RUN apt-get update && apt-get install -y \
    git \
    unzip \
    zip \
    libpq-dev \
    libonig-dev \
    libcurl4-gnutls-dev \
    libxml2-dev \
    && docker-php-ext-install pdo pdo_mysql bcmath mbstring xml

RUN pecl install redis \
    && docker-php-ext-enable redis

COPY --from=composer:latest /usr/bin/composer /usr/bin/composer

WORKDIR /var/www

COPY .env.example .env
COPY . .

ENTRYPOINT ["environment/dev/php/entrypoint.sh"]
CMD ["php-fpm"]
\end{lstlisting}

W listingu \ref{lst:dockerfile} obrazem bazowym jest php w wersji 8.2-fpm. Wyznacza się go za pomocą komendy FROM.

W późniejszych etap pracuje się na wybranym obrazie, modyfikując go w dowolny sposób.

W tym pliku kolejno:
\begin{enumerate}
    \item Aktualizowane i instalowane są wymagane przez \verb|Composer| biblioteki.
    \item Instalowane jest rozszerzenie Redis, po czym zostaje aktywowany.
    \item Kopiowany jest obraz najnowszej wersji \verb|Composera| do tworzonego obrazu PHP.
    \item Ustawiany jest katalog roboczy, dzięki czemu kolejne operacje będą w nim wykonywane.
    \item Kopiowany jest plik .env.example do pliku .env (w pliku tym są wszystkie zmienne projektu)
    \item Kopiowana jest cała zawartość programu.
    \item Uruchamiany jest skrypt \verb|entrypoint.sh|
    \item Uruchamiany jest proces php-fpm, który będzie nasłuchwiał i obsługiwał żądania.
\end{enumerate}

\subsection{Skrypt Entrypoint}
W projekcie, w celu instalacji \verb|composera| i wstępnej inicjalizacji plików, uzyty został skrypt w języku \verb|bash|.

\newpage
\begin{lstlisting}[language=bash, caption=Skrypt entrypoint.sh, label={lst:entrypoint}, numbers=left, frame=single]
#!/bin/bash

if [ ! -f .env ]; then
    cp .env.example .env
fi

if [ ! -d "vendor" ]; then
    composer install --no-dev --no-interaction --optimize-autoloader
fi

php artisan key:generate --force
php artisan migrate --force

exec php-fpm
\end{lstlisting}

Skrypt z listingu \ref{lst:entrypoint} wykonuje kilke kilka poleceń konsolowych w obrazie tworzonym w listingu \ref{lst:dockerfile}. Na początku kopiowany jest plik ze zmiennymi, ponieważ \verb|Laravel| potrzebuje niektórych danych w nim zawarych. Następnie instalowany jest composer.
Po instalacji generowany jest unikalny klucz szyfrujący dla aplikacji Laravel, który zostaje zapisany w pliku \verb|.env|. Następnie baza zostaje zmigrowana. Od teraz kontener PHP jest skofigurowany do pracy z frameworkiem \verb|Laravel|.