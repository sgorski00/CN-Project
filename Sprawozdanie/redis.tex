\section{Redis}
\verb|Redis| to system bazodanowy \verb|no-sql|, który działa w pamięci i przechowuje dane tymczasowe. Dzięki tej technologii zwiększa się wydajność aplikacji. Redis zostanie zintegrowany z kontenerem PHP i wspomoże dostęp do często używanych danych. Werjsa użyta w projekcie to \verb|Alpine| - najnowsza, lżejsza wersja technologii.

\subsection{docker-compose.yml}
\begin{lstlisting}[language=yaml, caption={Konfiguracja kontenera Redis w pliku docker-compose.yml}, label={lst:docker_compose_redis}, numbers=left, frame=single]
    redis:
        image: redis:alpine
        container_name: redis-container
        command: redis-server --appendonly yes
        ports:
            - "6379:6379"
        networks:
            - cn-network
\end{lstlisting}

W listingu \ref{lst:docker_compose_redis} widać konfigurację technologii \verb|Redis|. Nowością tutaj jest klucz \verb|command| z wartością \verb|redis-server --appendonly yes|. Klucz ten wywołuje komendę wpisaną w wartości poczas uruchomienia kontenera. Wartość \verb|redis-server| uruchamia serwer \verb|Redis|. Parametr \verb|--appendonly yes| włącza tryb zapisywania operacji do pliku.